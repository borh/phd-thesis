% Table of Contents - List of Tables/Figures/Listings and Acronyms

\refstepcounter{dummy}

\pdfbookmark[1]{\contentsname}{tableofcontents} % Bookmark name visible in a PDF viewer

\setcounter{tocdepth}{2} % Depth of sections to include in the table of contents - currently up to subsections

\setcounter{secnumdepth}{3} % Depth of sections to number in the text itself - currently up to subsubsections

\manualmark
\markboth{\spacedlowsmallcaps{\contentsname}}{\spacedlowsmallcaps{\contentsname}}
\tableofcontents
\automark[section]{chapter}
\renewcommand{\chaptermark}[1]{\markboth{\spacedlowsmallcaps{#1}}{\spacedlowsmallcaps{#1}}}
\renewcommand{\sectionmark}[1]{\markright{\thesection\enspace\spacedlowsmallcaps{#1}}}

\clearpage

\begingroup
\let\clearpage\relax
\let\cleardoublepage\relax
\let\cleardoublepage\relax

%----------------------------------------------------------------------------------------
%	List of Figures
%----------------------------------------------------------------------------------------

\refstepcounter{dummy}
%\addcontentsline{toc}{chapter}{\listfigurename} % Uncomment if you would like the list of figures to appear in the table of contents
\pdfbookmark[1]{\listfigurename}{lof} % Bookmark name visible in a PDF viewer

\listoffigures

\vspace*{8ex}
\newpage

%----------------------------------------------------------------------------------------
%	List of Tables
%----------------------------------------------------------------------------------------

\refstepcounter{dummy}
%\addcontentsline{toc}{chapter}{\listtablename} % Uncomment if you would like the list of tables to appear in the table of contents
\pdfbookmark[1]{\listtablename}{lot} % Bookmark name visible in a PDF viewer

\listoftables

\vspace*{8ex}
\newpage

% DEBUG: this shows source code listings in Japanese...
% %----------------------------------------------------------------------------------------
% %	List of Listings
% %----------------------------------------------------------------------------------------
%
% \refstepcounter{dummy}
% %\addcontentsline{toc}{chapter}{\lstlistlistingname} % Uncomment if you would like the list of listings to appear in the table of contents
% \pdfbookmark[1]{\lstlistlistingname}{lol} % Bookmark name visible in a PDF viewer
%
% \lstlistoflistings
%
% \vspace*{8ex}
% \newpage

%----------------------------------------------------------------------------------------
%	Acronyms
%----------------------------------------------------------------------------------------

\refstepcounter{dummy}
%\addcontentsline{toc}{chapter}{Acronyms} % Uncomment if you would like the acronyms to appear in the table of contents
\pdfbookmark[1]{Acronyms}{acronyms} % Bookmark name visible in a PDF viewer

\markboth{\spacedlowsmallcaps{Acronyms}}{\spacedlowsmallcaps{Acronyms}}

\chapter*{Acronyms}

\begin{acronym}[UML]
\acro{AdjN}{Adjective-Noun}
\acro{API}{Application Programming Interface}
\acro{BCCWJ}{Balanced Corpus of Contemporary Written Japanese}
\acro{CALL}{Computer-Assisted Language Learning}
\acro{CI}{Confidence Interval}
\acro{JASSO}{The Japan Student Services Organization}
\acro{JLPT}{Japanese Language Proficiency Test}
\acro{KWIC}{KeyWord In Context}
\acro{L2}{Second Language}
\acro{LDA}{Latent Dirichlet Allocation}
\acro{LUW}{Long Unit Word}
\acro{MD}{Multi-Dimensional}
\acro{MEXT}{Japanese Ministry of Education, Culture, Sports, Science and Technology}
\acro{MVR}{Modifier Verb Ratio}
\acro{NAIST}{Nara Advanced Institute of Science and Technology}
\acro{NDC}{Nippon Decimal Classification}
\acro{NINJAL}{National Institute for Japanese Language and Linguistics}
\acro{NLP}{Natural Language Processing}
\acro{NPAdj}{Noun-Particle-Adjective}
\acro{NPV}{Noun-Particle-Verb}
\acro{ROC}{Receiver Operating Characteristic}
\acro{SFL}{Systemic Functional Linguistics}
\acro{STJC}{Scientific and Technical Japanese Corpus}
\acro{SVM}{Support Vector Machine}
\acro{SUW}{Short Unit Word}
\acro{XML}{eXtensible Markup Language}
\end{acronym}

\endgroup

\cleardoublepage
